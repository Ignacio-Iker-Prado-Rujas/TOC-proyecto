\documentclass[11pt, a4paper, spanish, openright, twoside]{book}
\usepackage[spanish, activeacute]{babel}
\usepackage[utf8]{inputenc}
\usepackage[top=2.5cm, bottom=2.5cm, outer=1.75cm, inner=1.75cm, heightrounded, marginparwidth=2.5cm, marginparsep=0.3cm]{geometry}	%márgenes empequeñecidos
%\usepackage[top=2.5cm, bottom=2.25cm, outer=2.75cm, inner=2.75cm, heightrounded, marginparwidth=2.5cm, marginparsep=0.3cm]{geometry}	%márgenes originalmente
\usepackage{dpg}
\usepackage{afvc}
%Figuras
\usepackage[vflt]{floatflt}		%Entorno float-figure
%\graphicspath{/Temas/Tema01/Imagenes}

%%%%%%		chapter's style		%%%%%%%%%%%%%%%%%%%

\renewcommand{\thepage}{\arabic{page}}% Arabic page numbers\fancyhead{}
\pagestyle{fancy}
\fancyfoot{}
\fancyhead[RO]{}	%encabezado de pares: nombre de la sección
\fancyfoot[LE,RO]{\thepage}	%abajo a izqda en pares, derecha en impares: numero de pagina
\fancyhead[LE]{\nouppercase{\leftmark}} %cuadro izquierdo de pagina par: parte y contador
\fancyfoot[CE]{Tecnología y Organización de Computadores} 
\fancyfoot[CO]{Doble Grado Informática-Matemáticas - Universidad Complutense}
\renewcommand{\footrulewidth}{0.4pt}
\renewcommand{\headrulewidth}{0.4pt}		% linea por debajo del encabezado
\renewcommand{\sectionmark}[1]{\markright{\textbf{\thesection. #1}}}	%negrita
\renewcommand{\labelitemi}{$\circ$} %Primer itemize con circunferencia vacia
\renewcommand{\labelitemii}{$\cdot$} %Segundo itemize con punto pequeño
\renewcommand*{\thesection}{\arabic{section}}	% Hace que no apareca el indice de capitulos y que comience en section, GRACIAS A RUBEN
\setlength{\leftmarginii}{0em} %Segundo itemize sin sangria
\setlength{\leftmarginiii}{1em} %Tercer itemize casi sin sangria
\renewcommand{\labelitemiii}{ }
\pagenumbering{roman}
\newcommand*{\PKT}{\hbox{P}\kern-2.5pt\lower3.5pt\hbox{\small{K}}\kern-2.8pt\hbox{T}\kern-2pt}	%PiKey Team en bonito


\begin{document} 
\title{\Huge{\textsc{Manual Jetpack}} \\
	\vspace{0.7cm}
	 \textsc{\Large{Tecnología y Organización de Computadores}} \\
	\includegraphics[scale=0.3]{ucm.pdf}}
\author{{\Large{PiKey Team-}} \PKT \ : \vspace{0.2cm} \\
	Jesús Aguirre Pemán \\
	 Enrique Ballesteros Horcajo \\
	 Mayra Alexandra Castrosqui Florián\\
	 Jaime Dan Porras Rhee \\
	 Ignacio Iker Prado Rujas}
\date{\Today}
\maketitle

\newpage
\mbox{}
\thispagestyle{empty}						% Hoja en blanco, sin numeros ni nada
\newpage


\tableofcontents 							%INDICE hipervinculado

\newpage
\mbox{}
\thispagestyle{empty}						% Hoja en blanco, sin numeros ni nada
\newpage

\pagenumbering{arabic}						% Pone el contador de paginas a 1 y ahora en numeros normales

\vspace{3cm}

%ESTE FICHERO DEBERÍA SER DE UNAS DOS CARAS APROX.______________________________________________________

\section{Ficheros y dependencias}
El módulo central de nuestro proyecto es \textit{choca.vhd}. Es ahí donde se gestionan las posiciones de Barry, el fondo, obstáculos y monedas. Sin embargo, para leer los datos de memoria utilizamos varias ROMs:
\begin{itemize}
\item \textit{rom\_rgb\_9b\_fondo.vhd}: Memoria ROM que contiene los datos del fondo del juego.
\item \textit{rom\_rgb\_9b\_game-over-negro.vhd}: Memoria ROM que contiene los datos del ''Game Over''.
\item \textit{rom\_rgb\_9b\_arboles.vhd}: Memoria ROM que contiene los datos del fondo del tercer nivel del juego.
\item \textit{rom\_rgb\_9b\_barryair.vhd}: Memoria ROM que contiene los datos de Barry.
\item \textit{rom\_rgb\_9b\_flappy-nivelBW.vhd}: Memoria ROM que contiene los datos de los obstáculos del segundo nivel del juego.
\item \textit{rom\_rgb\_9b\_lab.vhd}: Memoria ROM que contiene los datos del fondo de los dos primeros niveles del juego.
\item \textit{rom\_rgb\_9b\_mapa\_facil.vhd}: Memoria ROM que contiene los obstáculos del primer nivel del juego.
\item \textit{rom\_rgb\_9b\_nivelfuegoBW.vhd}: Memoria ROM que contiene los obstáculos del tercer nivel del juego.
\item \textit{rom\_rgb\_9b\_nubes.vhd}: Memoria ROM que contiene el fondo del tercer nivel del juego.
\end{itemize}

Además, utilizamos distintas frecuencias para Barry, el fondo y los obstáculos, pues cada uno de ellos se mueve a una velocidad diferente. Para ello tenemos varios ficheros divisores:
\begin{itemize}
\item \textit{divisor.vhd}: 
\item \textit{divisor\_choques}: 
\item \textit{divisor\_movimiento\_fondo.vhd}: Fichero divisor que controla la velocidad de movimiento del fondo.
\item \textit{divisor\_movimiento\_inter\_fondo.vhd}: Fichero divisor que controla la velocidad de movimiento del fondo intermedio(para el último nivel).
\item \textit{divisor\_movimiento\_obstaculos.vhd}: Fichero divisor que controla la velocidad de movimiento de los obstáculos.
\item \textit{divisor\_movimiento\_moneda.vhd}: Fichero divisor que controla la velocidad de movimiento de las monedas.
\item \textit{divisor\_munyeco.vhd}: Fichero divisor que controla la velocidad de Barry.
\item \textit{divisor\_corre.vhd}: 
\end{itemize}

Con el fichero \textit{contador.vhd} controlaremos el número de monedas cogidas, que se mostrarán en los displays 7 segmentos de la placa mediante el fichero \textit{pines.ucf}.

Mediante el ficher \textit{control\_teclado.vhd} controlamos las teclas que pulsamos: la barra espaciadora para hacer ascender a Barry, y la 'p' para pausar la partida.

El randomgenerator no se si se usa ¿¿??


\section{nombre.bit}


\section{Pines y significado}
Conectaremos el teclado con los pines:

NET PS2CLK LOC = B16;
NET PS2DATA LOC = E13;
NET PS2CLK CLOCK\_DEDICATED\_ROUTE = FALSE;

Mostraremos el número de monedas mediante los displays de la placa extendida:
\begin{itemize}
\item placa izquierda:

NET displayizq<0> LOC=H14;
NET displayizq<1> LOC=M4;
NET displayizq<2> LOC=P1;
NET displayizq<3> LOC=N3;
NET displayizq<4> LOC=M15;
NET displayizq<5> LOC=H13;
NET displayizq<6> LOC=G16;

\item placa derecha:

NET displaydcha<0> loc=E2;
NET displaydcha<1> loc=E1;
NET displaydcha<2> loc=F3;
NET displaydcha<3> loc=F2;
NET displaydcha<4> loc=G4;
NET displaydcha<5> loc=G3;
NET displaydcha<6> loc=G1;
\end{itemize}

Por último, conectaremos los colores y la posición en la pantalla:

NET rgb<0> LOC=C9; \# BLU0;
NET rgb<1> LOC=E7; \# BLU1;
NET rgb<2> LOC=D5; \# BLU2;
NET rgb<3> LOC=A8; \# GRN0; 
NET rgb<4> LOC=A5; \#  GRN1;
NET rgb<5> LOC=C3; \# GRN2;
NET rgb<6> LOC=C8; \# RED0;
NET rgb<7> LOC=D6; \# RED1;
NET rgb<8> LOC=B1; \# RED2;
NET hsyncb LOC=B7; \# HSYNC \# 
NET vsyncb LOC=D8; \# VSYNC \# 

\section{Otros aquelarres de inicialización}
Qué carajos hay que poner aquí??
El aquelarre o sabbat es una reunión nocturna de brujas y brujos presidida por Satanás que generalmente se presenta en forma de macho cabrío. Fuente: wikipedia.

\section{Cómo se ``juega''}
Tras el despegue, simplemente pulsa la barra espaciadora para ascender y deja de pulsarla para descender.  El objetivo es esquivar todos los obstáculos, y recoger el mayor número de monedas posible. A medida que recojas monedas, el juego cambiará de nivel.
Si deseas pausar el juego debes pulsar la tecla `P'.\\
Si deseas salir de la pausa para seguir jugando debes pulsar la barra espaciadora. \\
Al chocar con un obstáculo el juego se parará mostrando un mensaje de Game Over.  Si pulsamos la barra espaciadora el juego comenzará de nuevo. \footnote{En el video se puede observar que después de haber chocado con un obstáculo, al pulsar el espacio se puede seguir jugando normalmente. En este caso está puesto así a propósito ya que de este modo era más fácil grabar el video de la presentación de cada nivel.}\\
Si deseamos empezar el juego desde el principio debemos activar la señal de reset y desactivarla tras un instante. 



\section{Qué espero ver}
El juego consta de 3 niveles diferentes:
\begin{itemize}
\item \textbf{Nivel 1: obstáculos espaciales y fondo de laboratorio.}
\item \textbf{Nivel 2: tuberías como obstáculos y fondo de laboratorio.}
\item \textbf{Nivel 2: obstáculos de fuego y fondo inspirado en Super Mario.}
Empezamos en el nivel 1, y al conseguir 5 monedas pasaremos al nivel 2. Cuando obtengamos 5 monedas más, avanzaremos al nivel 3, y recogiendo otras 3 monedas volveremos al nivel 1, y así sucesivamente hasta llegar a completar el desafío: 60 monedas.
\end{itemize}
La versión para FPGA del juego Jetpack Joyride. 


\end{document}